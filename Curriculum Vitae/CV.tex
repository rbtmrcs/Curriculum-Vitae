%%%%%%%%%%%%%%%%%
% This is an sample CV template created using altacv.cls
% (v1.1.5, 1 December 2018) written by LianTze Lim (liantze@gmail.com). Now compiles with pdfLaTeX, XeLaTeX and LuaLaTeX.
%
%% It may be distributed and/or modified under the
%% conditions of the LaTeX Project Public License, either version 1.3
%% of this license or (at your option) any later version.
%% The latest version of this license is in
%%    http://www.latex-project.org/lppl.txt
%% and version 1.3 or later is part of all distributions of LaTeX
%% version 2003/12/01 or later.
%%%%%%%%%%%%%%%%

%% If you need to pass whatever options to xcolor
\PassOptionsToPackage{dvipsnames}{xcolor}

%% If you are using \orcid or academicons
%% icons, make sure you have the academicons
%% option here, and compile with XeLaTeX
%% or LuaLaTeX.
% \documentclass[10pt,a4paper,academicons]{altacv}

%% Use the "normalphoto" option if you want a normal photo instead of cropped to a circle
% \documentclass[10pt,a4paper,normalphoto]{altacv}

\documentclass[10pt,a4paper,ragged2e,table]{altacv}

\usepackage{minipage-marginpar}
\usepackage{multicol}

%% AltaCV uses the fontawesome and academicon fonts
%% and packages.
%% See texdoc.net/pkg/fontawecome and http://texdoc.net/pkg/academicons for full list of symbols. You MUST compile with XeLaTeX or LuaLaTeX if you want to use academicons.

% Change the page layout if you need to
\geometry{left=1cm,right=9cm,marginparwidth=6.8cm,marginparsep=1.2cm,top=1.25cm,bottom=1.25cm}

% Change the font if you want to, depending on whether
% you're using pdflatex or xelatex/lualatex
\ifxetexorluatex
  % If using xelatex or lualatex:
  \setmainfont{Carlito}
\else
  % If using pdflatex:
  \usepackage[utf8]{inputenc}
  \usepackage[T1]{fontenc}
  \usepackage[default]{lato}
\fi

% Change the colours if you want to
\definecolor{F_LPink1}{HTML}{ffd9ea}
\definecolor{F_LPink2}{HTML}{ffedf5}
\definecolor{F_Pink}{HTML}{FF7DB6}
\definecolor{F_DPink}{HTML}{9C5373}
\definecolor{DarkGrey}{HTML}{282828}
\definecolor{LightGrey}{HTML}{666666}

\colorlet{heading}{F_DPink}
\colorlet{accent}{F_Pink}
\colorlet{emphasis}{DarkGrey}
\colorlet{body}{LightGrey}

% Change the bullets for itemize and rating marker
% for \cvskill if you want to
\renewcommand{\itemmarker}{{\small\textbullet}}
\renewcommand{\ratingmarker}{\faCircle}

\begin{document}
\name{Tyler Wright}
\tagline{Maths and Computer Science (BSc) student at the University of Bristol}
%\photo{2cm}{logo}
\personalinfo{%
  % Not all of these are required!
  % You can add your own with \printinfo{symbol}{detail}
  \email{contact@fluxanoia.co.uk}
  \homepage{fluxanoia.co.uk}
  \github{github.com/Fluxanoia}
  \location{Bristol, United Kingdom}
  %% You MUST add the academicons option to \documentclass, then compile with LuaLaTeX or XeLaTeX, if you want to use \orcid or other academicons commands.
  % \orcid{orcid.org/0000-0000-0000-0000}
}

%% Make the header extend all the way to the right, if you want.
\begin{fullwidth}
\makecvheader
\end{fullwidth}

%% Depending on your tastes, you may want to make fonts of itemize environments slightly smaller
% \AtBeginEnvironment{itemize}{\small}

%% Provide the file name containing the sidebar contents as an optional parameter to \cvsection.
%% You can always just use \marginpar{...} if you do
%% not need to align the top of the contents to any
%% \cvsection title in the "main" bar.
\cvsection[page1sidebar]{Education}

\cvevent{University of Bristol}{Maths and Computer Science (BSc)}{2018 -- 2021}{}
In first year, I achieved a mark of 83\% overall. I am now in 
second year, taking: \begin{multicols}{2}
  \begin{itemize}
    \item Data Structures and Algorithms,
    \item Theory of Computation,
    \item Symbols, Patterns, and Symbols,
    \item Databases and Cloud Concepts,
  \end{itemize}
  \columnbreak
  \begin{itemize}
    \item Language Engineering,
    \item Linear Algebra 2,
    \item Combinatorics.
  \end{itemize}
\end{multicols}

\divider

\cvevent{King's College London Mathematics School}{A Level}{2016 -- 2018}{}

\noindent
\begin{minipage}[c]{0.6\linewidth}

  \renewcommand{\arraystretch}{1.5}
  \begin{tabular}{ c c c }
    \rowcolor{F_LPink1} Grade & Type & Subject \\
    \rowcolor{F_LPink2} A* & A2 & Further Maths \\
    \rowcolor{F_LPink2} A* & A2 & Maths \\
    \rowcolor{F_LPink2} A & A2 & Physics \\
    \rowcolor{F_LPink2} A & AS & Computer Science
  \end{tabular}

\end{minipage}
\begin{minipage}[c]{0.3\linewidth}

  At King's Maths School I took part in many 
  extra-curricular projects including a poisson 
  simulator via the binomial in Ruby and an essay on infinity 
  and its types.

\end{minipage}

\cvsection{Experience}

\cvevent{Founder and President of the Maths and Computer Science Society}{University of Bristol}{October 2019, Present}{Bristol}
I coordinate my committee to run events, talks, socials, and support for the
Maths and Computer Science students of Bristol University.

\divider

\cvevent{Open Day Ambassador}{University of Bristol}{June 2019, September 2019}{Bristol}
I helped direct and inform visitors about the university, sharing what
I personally enjoyed about Bristol, maths, and computer science.

\divider

\cvevent{Student Ambassador}{King's College London}{July 2018, July 2019}{London}
I was a teaching assistant in post-GCSE level maths and physics classes,
helping to manage all students safety and wellbeing whilst giving
them an intuitive insight into learning at A Level and higher education.

\clearpage

%% If the NEXT page doesn't start with a \cvsection but you'd
%% still like to add a sidebar, then use this command on THIS
%% page to add it. The optional argument lets you pull up the
%% sidebar a bit so that it looks aligned with the top of the
%% main column.
% \addnextpagesidebar[-1ex]{page3sidebar}


\end{document}
