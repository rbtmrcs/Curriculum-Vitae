%%%%%%%%%%%%%%%%%
% This is an sample CV template created using altacv.cls
% (v1.1.5, 1 December 2018) written by LianTze Lim (liantze@gmail.com). Now compiles with pdfLaTeX, XeLaTeX and LuaLaTeX.
%
%% It may be distributed and/or modified under the
%% conditions of the LaTeX Project Public License, either version 1.3
%% of this license or (at your option) any later version.
%% The latest version of this license is in
%%    http://www.latex-project.org/lppl.txt
%% and version 1.3 or later is part of all distributions of LaTeX
%% version 2003/12/01 or later.
%%%%%%%%%%%%%%%%

%% If you need to pass whatever options to xcolor
\PassOptionsToPackage{dvipsnames}{xcolor}

%% If you are using \orcid or academicons
%% icons, make sure you have the academicons
%% option here, and compile with XeLaTeX
%% or LuaLaTeX.
% \documentclass[10pt,a4paper,academicons]{altacv}

%% Use the "normalphoto" option if you want a normal photo instead of cropped to a circle
% \documentclass[10pt,a4paper,normalphoto]{altacv}

\documentclass[10pt,a4paper,ragged2e,table]{altacv}

\usepackage{minipage-marginpar}

%% AltaCV uses the fontawesome and academicon fonts
%% and packages.
%% See texdoc.net/pkg/fontawecome and http://texdoc.net/pkg/academicons for full list of symbols. You MUST compile with XeLaTeX or LuaLaTeX if you want to use academicons.

% Change the page layout if you need to
\geometry{left=1cm,right=9cm,marginparwidth=6.8cm,marginparsep=1.2cm,top=1.25cm,bottom=1.25cm}

% Change the font if you want to, depending on whether
% you're using pdflatex or xelatex/lualatex
\ifxetexorluatex
  % If using xelatex or lualatex:
  \setmainfont{Carlito}
\else
  % If using pdflatex:
  \usepackage[utf8]{inputenc}
  \usepackage[T1]{fontenc}
  \usepackage[default]{lato}
\fi

% Change the colours if you want to
\definecolor{F_LPink1}{HTML}{ffd9ea}
\definecolor{F_LPink2}{HTML}{ffedf5}
\definecolor{F_Pink}{HTML}{FF7DB6}
\definecolor{F_DPink}{HTML}{9C5373}
\definecolor{DarkGrey}{HTML}{282828}
\definecolor{LightGrey}{HTML}{666666}

\colorlet{heading}{F_DPink}
\colorlet{accent}{F_Pink}
\colorlet{emphasis}{DarkGrey}
\colorlet{body}{LightGrey}

% Change the bullets for itemize and rating marker
% for \cvskill if you want to
\renewcommand{\itemmarker}{{\small\textbullet}}
\renewcommand{\ratingmarker}{\faCircle}

\begin{document}
\name{Tyler Wright}
\tagline{Cover Letter for < Company Name >}
%\photo{2cm}{logo}
\personalinfo{%
  % Not all of these are required!
  % You can add your own with \printinfo{symbol}{detail}
  \email{contact@fluxanoia.co.uk}
  \homepage{fluxanoia.co.uk}
  \github{github.com/Fluxanoia}
  \location{Bristol, United Kingdom}
  %% You MUST add the academicons option to \documentclass, then compile with LuaLaTeX or XeLaTeX, if you want to use \orcid or other academicons commands.
  % \orcid{orcid.org/0000-0000-0000-0000}
}

%% Make the header extend all the way to the right, if you want.
\begin{fullwidth}
\makecvheader

\AtBeginEnvironment{itemize}{\small}

%% Provide the file name containing the sidebar contents as an optional parameter to \cvsection.
%% You can always just use \marginpar{...} if you do
%% not need to align the top of the contents to any
%% \cvsection title in the "main" bar.
\cvsection{About}

< Text on how my background will benefit the work experience >

\cvsection{Education}

< Text on how my education will benefit the work experience >
\newline
< Text on how the work experience will benefit my education/development >

\cvsection{Experience}

< Text on how my experience will benefit the work experience >
\newline
< Text on how the experience will benefit my work experience >

\clearpage

\end{fullwidth}

\end{document}
